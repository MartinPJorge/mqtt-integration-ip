\documentclass{upmassignment}
\usepackage[spanish]{babel}
\usepackage{ifthen}
\usepackage{amsmath}
\usepackage{amsfonts}
\usepackage{booktabs}
\usepackage[x11names]{xcolor}
\usepackage{tcolorbox}
\usepackage{cclicenses}
\usepackage{url}

\usepackage{listings}
\lstset{basicstyle=\ttfamily,
  showstringspaces=false,
  commentstyle=\color{red},
  keywordstyle=\color{blue},
  backgroundcolor=\color{gray!30},
}


\usetikzlibrary{calc}



% Para mostrar/ocultar soluciones
\newboolean{show}
\setboolean{show}{true}
\setboolean{show}{false}
\usepackage{environ}
\NewEnviron{solucion}{
  \ifshow
      \begin{answer}\BODY\end{answer}
  \fi}






\coursetitle{Redes y Servicios}
\courselabel{RSER}
\exercisesheet{Integración}{}
\student{\ }%
\semester{Primer Semestre 2024/2025}
\date{\today}
\university{Universidad Politécnica de Madrid}
\school{Departamento de Ingeniería de Sistemas Telemáticos}
%\usepackage[pdftex]{graphicx}
%\usepackage{subfigure}


\setlength{\textwidth}{5.0in}
\linespread{1.3}
\renewcommand{\PB}{{\bfseries Problema}}



\begin{document}



\section*{Introducción}

En esta práctica\footnote{Este material está protegido por la
licencia
CC BY-NC-SA 4.0.
\byncsa
} se integrarán
todas las componentes MQTT en el
siguiente escenario de red:

\begin{figure}[h]
    \centering
    \begin{tikzpicture}

        % PHY/VIR components
        \node at (0,0)
            (client)
            {\includegraphics[width=1cm]{figs/pc.png}};
        \node at (client.north)
            {h2};

        \node at ($(client)+(4,0)$)
            (router1)
            {\includegraphics[width=1cm]{figs/router.png}};
        \node at
            ($(router1.north)+(0,.2)$)
            {r1};

        \node at ($(router1)-(0,2.5)$)
            (router2)
            {\includegraphics[width=1cm]{figs/router.png}};
        \node at (router2.north east)
            {r2};

        
        \node at ($(router1)+(5,0)$)
            (broker)
            {\includegraphics[width=1cm]{figs/pc.png}};
        \node at (broker.north)
            {host};


        \node at ($(router2)-(0,2.75)$)
            (subs)
            {\includegraphics[width=1cm]{figs/pc.png}};
        \node at (subs.south)
            {h3};
        

        % Links
        \draw (client) -- (router1.west);
        \draw (broker) -- (router1.east);
        \draw (router2.north) -- (router1.south);
        \draw (router2.south) -- (subs.north);

        % Interfaces
        % h2
        \node[anchor=west]
            at ($(client.east)+(0,.2)$)
            {eth1};
        % r1
        \node[anchor=east]
            at ($(router1.west)+(0,.2)$)
            {eth1};
        \node[anchor=north west]
            at (router1.south)
            {eth2};
        \node[anchor=west]
            at ($(router1.east)+(0,.2)$)
            {eth3};
        % host
        \node[anchor=east]
            at ($(broker.west)+(0,.2)$)
            {Net3-e00};
        % r2
        \node[anchor=east]
            at ($(router2.north)+(0,.2)$)
            {eth1};
        \node[anchor=north east]
            at (router2.south)
            {eth2};
        % h3
        \node[anchor=south east]
            at (subs.north)
            {eth1};

        % COmponentes - h2
        \node[draw,fill=gray!20]
            (clientpy)
            at ($(client.south)-(0,.25)$)
            {\texttt{client.py}};
        \node[draw,fill=gray!20,
            anchor=north]
            (utils)
            at (clientpy.south)
            {\texttt{utils.py}};
        \node[draw,fill=gray!20,
            anchor=north]
            (reporter)
            at (utils.south)
            {\texttt{reporter.sh}};

        % COmponentes - host
        \node[draw,fill=gray!20,align=center]
            (brokeremqx)
            at ($(broker.south)-(0,.5)$)
            {\texttt{broker}\\\texttt{emqx}};
        \node[draw,fill=gray!20,align=center,
            anchor=north]
            (datasource)
            at ($(brokeremqx.south)$)
            {datasource};
        \node[draw,fill=gray!20,align=center,
            anchor=north]
            (grafana)
            at ($(datasource.south)$)
            {grafana};

        % Componentes - h3
        \node[draw,fill=gray!20,align=center,
            anchor=north]
            (subspy)
            at ($(subs.east)+(1,0)$)
            {\texttt{subs.py}};
    \end{tikzpicture}
\end{figure}



\section*{Lanzar escenario}
En primer lugar debe arrancar la máquina
virtual utilizada en el curso.
Sírvase del comando de sesiones
anteriores.
Todas las indicaciones de ahora
en adelante asumen que está dentro
de la máquina virtual.

Descargue en la misma carpeta los ficheros
\texttt{install.sh} y
\texttt{vnx-integracion.xml}.
A continuación abra una terminal
y ejecute
\begin{lstlisting}[language=bash]
$ cd ~/Downloads
$ sudo ./install.sh
\end{lstlisting}
en caso de que haya descargado los
ficheros en la carpeta
\texttt{\~{}/Downloads}.
Espere unos 5\,\textrm{min} para que
termine de desplegarse todo el escenario.



\section*{Configuración de red}
\subsection*{Plan de direccionamiento IP}
En este apartado tendrá que configurar
las direcciones IP de las interfaces:
h2.\texttt{eth1},
r1.\texttt{eth1},
r1.\texttt{eth2},
r2.\texttt{eth1},
r2.\texttt{eth2},
h3.\texttt{eth1}.
Para ello tiene que decidir qué prefijos
(por ejemplo \texttt{10.1.0.0/24})
tendrán las siguientes subredes:
\begin{itemize}
    \item Net0: compuesta por
        h2.\texttt{eth1} y
        r1.\texttt{eth1}.
    \item Net1: compuesta por
        r1.\texttt{eth2} y
        r2.\texttt{eth1}.
    \item Net2: compuesta por
        r2.\texttt{eth2} y
        h3.\texttt{eth1}.
\end{itemize}

Una vez haya definido los prefijos
de cada subred, decida qué direcciones
IP tendrá cada una de las interfaces que
hay en cada subred.
Para ello acceda a la consola del
dispositivo
(login: root, password: xxxx)
y ejecute
\begin{lstlisting}[language=bash]
$ ip address add IP/PREF dev IF
\end{lstlisting}
donde \texttt{IP/PREF} es la dirección
a asignar a la interfaz \texttt{IF}
y \texttt{PREF} es la longitud del prefijo.

\emph{Nota}: la red Net3 interconecta
su host (la máquina virtual)
con el router virtual r1.
Esta red ya viene configurada.


\subsection*{Reenvío IP}
En este apartado debe configurar las
tablas de direccionamiento de los
equipos h2,h3,r1 y r2.
Debe configurar el direccionamiento
para que tenga \texttt{ping}
entre todos los dispositivos del escenario.

Para ello utilice el siguiente comando
en cada una de las terminales de los equipos
\begin{lstlisting}[language=bash]
$ ip route add SUBNET via NEXTHOP
\end{lstlisting}
Con este comando especifica que para
llegar a la subred \texttt{SUBNET}
(por ejemplo, 10.1.0.0/24)
tiene que enviar el tráfico al
siguiente salto \texttt{NEXTHOP}
(por ejemplo, 10.1.0.2).


\emph{Nota}: recuerde que el broker
EMQX se ejecuta como un contenedor
dentro de la máquina virtual.
El contenedor suele desplegarse en la
subred \texttt{172.17.0.0/16}.
Por tanto, deberá añadir rutas en todos
los equipos del escenario para poder
alcanzar el dicha subred.







\section*{Servicio de Detección de
Anomalías}
Una vez tenga configurado el escenario
de red, tiene que arrancar el servicio
de detección de anomalías que ha
desarrollado.

En concreto, tiene que ejecutar:
\begin{itemize}
    \item el \texttt{client.py}
        (sirviéndose del
        \texttt{utils.py} y
        \texttt{reporter.sh})
        en
        el h2;
    \item el \texttt{subs.py}
        en el h3; y
    \item el broker EMQX, grafana,
        y el datasource en el
        host (la máquina virtual).
\end{itemize}

\emph{Nota}: todos los archivos que
tenga en la carpeta
\texttt{/home/upm/shared}
de su máquina virtual están
accesibles en la carpeta
\texttt{/shared} de cada componente
del escenario.
Se recomienda tener todos los scripts
en la carpeta
\texttt{/home/upm/shared} para
facilitarle la edición y ejecución
en cada componente.


\subsection*{Arrancar broker EMQX, datasource
y Grafana}
Lo primero que va a realizar es
arrancar el contenedor del broker EMQX
y Grafana (junto al plugin del datasource).
Para ello, acceda al host (la máquina
virtual) y ejecute los comandos
ya utilizados en prácticas anteriores:
\begin{lstlisting}[language=bash]
$ docker run -d --name emqx -p 1883:1883 -p 8083:8083\
  -p 8883:8883 -p 8084:8084 -p 18083:18083\
  emqx/emqx:5.0.2

$ sudo docker run -d -p 3000:3000 --name=grafana \
  -e "GF_INSTALL_PLUGINS=grafana-mqtt-datasource" \
  grafana/grafana-enterprise
\end{lstlisting}
puede comprobar que todo se ha arrancado
correctamente si ve dos contenedores
al ejecutar
\begin{lstlisting}[language=bash]
$ docker ps
\end{lstlisting}

Para comprobar que h2 y h3 pueden
acceder al broker,
ejecute un \texttt{ping}
al broker EMQX.
Puede ver la dirección IP del broker
EMQX usando el siguiente comando:
\begin{lstlisting}[language=bash]
$ docker inspect `docker ps | grep 5.0.2\
  | cut -d' ' -f1 ` | grep "IPAddress"
\end{lstlisting}



\subsection*{Arrancar el cliente}
Una vez haya comprobado que hay
conexión entre todos los equipos
vamos a comenzar a generar métricas en h2
con el \texttt{reporter.sh}.
Para ello ejecute el siguiente comando
en la terminal
\begin{lstlisting}[language=bash]
$ sudo lxc-attach h2 -- cd /shared/DIR; ./reporter.sh
\end{lstlisting}
donde \texttt{DIR} es la carpeta donde
se encuentran todos sus archivos.
Esto comenzará a generar el
\texttt{report.csv} en la carpeta 
correspondiente.

Acto seguido, vamos a ejecutar nuestro
\texttt{client.py}. Acceda a la terminal
h2 y ejecute el cliente actualizando
la dirección IP del broker.
Si la configuración de red es correcta,
debería ver cómo el broker va recibiendo
mensajes.

\emph{Note}: compruebe que los
\texttt{PUBLISH} envían un JSON con
las métricas de cada línea de reporte.


\subsection*{Visualización de métricas}
En la máquina virtual, acceda a grafana
y siga los pasos de prácticas anteriores
para visualizar cómo va evolucionando
la temperatura.


\subsection*{Arrancar el
suscriptor/detector de anomalías}
El programa \texttt{subs.py} lo vamos
a ejecutar en la consola de h3.
Para ello, acceda a la carpeta
\texttt{/shared/DIR} y ejecute el
script. Recuerde que la dirección IP del
broker debe corresponder con la del
contenedor corriendo en el escenario.

Si la configuración de red es correcta,
debería recibir \texttt{PUBLISH}
desde el broker y en pantalla deberían
mostrarse las alertas de su script.



\section*{Evaluación}
La evaluación de la integración
se llevará a cabo por medio de un examen
práctico en el que todas las personas del
grupo deben demostrar que:
\begin{itemize}
    \item saben configurar direcciones IP;
    \item saben configurar una tabla
        de direccionamiento;
    \item comprenden la interconectividad
        IP del escenario;
    \item conocen la mensajería MQTT
        intercambiada en el escenario;
    \item conocen las distintas QoS disponibles
        en MQTT;
    \item saben transformar los datos en
        Grafana para visualizar la
        serie temporal;
    \item comprenden el funcionamiento
        de su \texttt{client.py}
        y \texttt{subs.py};
    \item saben analizar trazas de
        Wireshark correspondientes al
        intercambio de mensajes MQTT; y
    \item pueden desplegar el escenario
        de integración.
\end{itemize}



\begin{tcolorbox}
    \textbf{Atención I}: se recomienda,
    a modo de ensayo, que cada componente
    sea capaz de realizar la integración
    por si misma.

    \textbf{Atención II}: se recomienda,
    a modo de ensayo, realizar capturas
    en wireshark en la interfaz
    r1-eth3. Esta captura le mostrará
    todos los mensajes MQTT de su servicio.
\end{tcolorbox}








% \section*{Introducción}
% 
% \noindent
% En esta práctica\footnote{Todas las preguntas
% tienen el mismo valor/puntuación.
% Cada apartado del Problema~2 tiene el
% mismo valor que el resto de Problemas.}\footnote{Este material está protegido por la
% licencia
% CC BY-NC-SA 4.0.
% \byncsa
% }
% vamos a crear un cliente
% MQTT
% \texttt{client.py}
% que enviará constantes vitales
% de un Sensor.
% Para ello se proporciona el programa
% \texttt{reporter.sh}
% (disponible en Moodle)
% que va añadiendo líneas a un archivo
% \texttt{report.csv}
% que almacena las constantes vitales
% de un sensor.
% 
% 
% 
% 
% 
% 
% \begin{figure}[h]
%     \centering
% \begin{tikzpicture}
%     \node[draw,rectangle] (sensor) at (0,0)
%         {Sensor};
%     \node[draw,rectangle] (script)
%         at ($(sensor)+(0,-1)$)
%         {\texttt{reporter.sh}};
%     \node[draw,rectangle,anchor=west]
%         (report)
%         at ($(script.east)+(1,0)$)
%         {\texttt{report.csv}};
%     \node[draw,rectangle]
%         (utils)
%         at ($(report)+(0,1)$)
%         {\texttt{utils.py}};
%     \node[draw,rectangle,fill=gray!30]
%         (client)
%         at ($(utils)+(0,1)$)
%         {\texttt{client.py}};
%     \node[draw,rectangle]
%         (server)
%         at ($(client)+(5,0)$)
%         {server};
%     \node[align=center,anchor=north]
%         at (server.south)
%         {\texttt{broker.emqx.io}\\
%         \texttt{port:1883}};
% 
%     \draw[arrows=<->] (sensor.south)
%         -- (script.north);
%     \draw[arrows=->] (script.east)
%         -- (report.west);
%     \draw[arrows=<->] (report.north)
%         -- (utils.south);
%     \draw[arrows=<->] (utils.north)
%         -- (client.south);
%     \draw[arrows=<->] (client.east)
%         --
%         node[midway,above,align=center]
%         {MQTT\\connection}
%         (server.west);
% 
% 
% 
% \end{tikzpicture}
% \caption{Escenario de la práctica.}
% \label{fig:escenario}
% \end{figure}
% 
% 
% \section*{Instalación de Dependencias}
% \noindent
% Antes de comenzar la práctica debemos
% instalar las dependencias necesarias
% para python y MQTT. 
% Descargue el fichero
% \texttt{practica-cliente.zip}
% de Moodle y descomprímalo en
% su carpeta personal.
% Abra una terminal y ejecute
% las siguientes líneas
% \begin{lstlisting}[language=bash]
% $ cd ~/practica-cliente
% $ ./dependencies.sh
% \end{lstlisting}
% 
% 
% 
% 
% 
% \section*{Generación de Constantes Vitales}
% \noindent Para generar el archivo
% \texttt{report.csv} con las constantes
% vitales, ejecute el programa
% que reporta las medidas del sensor:
% \begin{lstlisting}[language=bash]
% $ cd ~/practica-cliente
% $ ./reporter.sh
% \end{lstlisting}
% Tras ejecutar las líneas de arriba,
% verá que se ha generado un archivo
% \texttt{report.csv} con las siguientes
% columnas (separadas por comas):
% \begin{lstlisting}
% Idx, Time (s), HR (BPM), RESP (BPM), SpO2 (%),TEMP (*C),OUTPUT
% 0,0,94,21,97,36.2,Normal
% 1,1,94,25,97,36.2,Normal
% 2,2,101,25,93,38,Abnormal
% 3,3,55,11,100,35,Abnormal
% \end{lstlisting}
% \emph{Nota}:
% el archivo \texttt{report.csv} se genera
% de cero para cada ejecución del
% el \texttt{reporter.sh}. 
% 
% 
% \section*{Lectura de Constantes Vitales}
% \noindent 
% A continuación va a programar la
% función \texttt{read\_report()} encargada
% de leer constantes vitales del reporte.
% Esta función se encuentra en el archivo
% \texttt{utils.py} y se encarga
% de leer el archivo de reporte
% \texttt{report.csv}. Por ejemplo,
% podemos empezar en la tercera línea
% del archivo y leer dos líneas usando
% \begin{lstlisting}[language=bash]
% $ python3
% >>> import utils
% >>> utils.read_report('report.csv',10,2)
% ['9,9,94,26,97,29,Normal\n', '10,10,94,26,97,42,Abnormal\n']
% \end{lstlisting}
% 
% Para más detalles sobre el funcionamiento
% de \texttt{read\_report()}, lea
% la descripción de la función en el
% archivo \texttt{utils.py}.
% Tendrá que utilizar las funciones
% \texttt{open()} y \texttt{read\_lines()}
% para programar la función.
% 
% 
% \begin{problemlist}
%     \pbitem Programe la función
%     \texttt{read\_report()} y
%     ejecútela usando como parámetros
%     \texttt{LAST\_LINE=$X$}
%     y
%     \texttt{num\_lines=$\lceil \tfrac{X}{2} \rceil$},
%     donde $X$ es el número de grupo
%     asignado por el profesor.
%     Copie el resultado en el
%     campo \texttt{lectura} del
%     archivo
%     \texttt{respuestas-X.json}
%     usando dobles comillas:
% \begin{lstlisting}
% {
%     "lectura": [ "9,9,94,26,97,29,Normal\n",
%       "10,10,94,26,97,42,Abnormal\n"]
% }
% \end{lstlisting}
% \end{problemlist}
% \emph{Nota}: deje corriendo
% \texttt{reporter.sh} para que genere
% suficientes reportes.
% 
% 
% 
% \section*{Cliente MQTT}
% \subsection*{Conexión}
% \noindent
% A continuación va a programar parte
% de la lógica del cliente MQTT.
% En concreto se va a programar la conexión
% del \texttt{client.py} con el
% server ilustrada en la
% Figura.~\ref{fig:escenario},
% donde se especifica la dirección y puerto
% del servidor.
% 
% Para la comunicación con el servidor
% mediante MQTT,
% el programa \texttt{client.py} utiliza
% la librería \texttt{PAHO} de python.
% Los pasos para abrir una conexión
% (consulte la plantilla del cliente
% \texttt{client.py})
% son los siguientes:
% \begin{enumerate}
%     \item Conectarse con el servidor
%         MQTT: función
%         \texttt{connect\_mqtt()}; e
%     \item invocar el bucle de publicación
%         de mensajes:
%         función \texttt{publish()}.
% \end{enumerate}
% 
% \begin{figure}[h]
%     \centering
%     \begin{tikzpicture}
%         \node[draw,rectangle,fill=gray!30]
%             (client) at (0,0)
%             {\texttt{client.py}};
%         \node[draw,rectangle] (server) at
%             ($(client)+(8,0)$)
%             {server};
%         \node[align=center,anchor=north]
%             at (server.south)
%             {\texttt{broker.emqx.io}\\
%             \texttt{port:1883}};
% 
%         \draw[arrows=<->]
%             (client.north east)
%             --
%             node[midway,above]
%             {\texttt{connect\_mqtt()}}
%             (server.north west);
%         \draw[arrows=->]
%             (client.south east)
%             --
%             node[midway,below]
%             {\texttt{publish()}}
%             (server.south west);
%     \end{tikzpicture}
%     \caption{Conexión y publicación del
%     cliente.}
%     \label{fig:connect-publish}
% \end{figure}
% 
% 
% Modifique la plantilla de
% \texttt{client.py} para conectarse
% al servidor y responda a las siguientes
% preguntas.
% 
% \begin{problemlist}
%     \stepcounter{enumi}
%     \pbitem Inicie una captura wireshark
%         en todas las interfaces
%         (\texttt{any}) poniendo filtro
%         ``mqtt''. Ejecute en la consola
%         \texttt{client.py} e identifique
%         las tramas de conexión y ACK
%         en wireshark. Detenga la captura
%         y rellene las
%         siguientes preguntas en el
%         JSON de respuestas:
%         \begin{enumerate}
%             \item ¿Cuál es el ID de su cliente?
%             \item En la captura wireshark,
%                 ¿cuál es el número de trama
%                 del ACK de la conexión?
%         \end{enumerate}
%         Guarde\footnote{Para guardas solo los paquetes MQTT
%                 basta con poner en el filtro
%                 ``mqtt'',
%                 pinchar en
%                 File-Export Specified Packets,
%                 y en el cuadro ``Packet
%                 Range'' pinchar en la
%                 bola  ``Displayed''.}
%                 la captura en el fichero
%         \texttt{captura-connect-grupoX.pcapng}.
% Siga el mismo procedimiento del
% pie de página para todas las capturas
% de la práctica.
% \end{problemlist}
% 
% 
% 
% \subsection*{Publicación}
% \noindent A continuación debe
% modificar el \texttt{client.py} para
% leer los datos del sensor
% (archivo \texttt{report.csv}) y publicarlos
% en el topic
% \texttt{rserGX/vitals}, donde X
% es su número de grupo.
% 
% La función \texttt{publish()}
% contiene un bucle que continuamente publica
% mensajes MQTT. Modifique el bucle
% para que espere
% $(X\mod{5}) + 5$ segundos al final de cada
% iteración.
% Además, modifique \texttt{publish()}
% para invocar a \texttt{read\_report()} y
% publicar \emph{una a una}
% las constantes vitales.
% %% En concreto,
% %% debe enviar el valor de
% %% la columna \texttt{Time (s)}
% %% (columna número 1)
% %% y la columna
% %% $(X \mod{4}) + 2$.
% 
% 
% 
% Tras haber realizado las modificaciones
% indicadas, inicie una captura en
% wireshark filtrando paquetes MQTT y
% responda a las siguientes
% preguntas usando la captura entregada.
% Para ello responda en
% en el JSON de respuestas a:
% 
% 
% \begin{problemlist}
%     \stepcounter{enumi}
%     \stepcounter{enumi}
%     \pbitem ¿En qué instante
%         se envía la muestra $X$?
%         (valor ``Time'' en wireshark).
% 
%     \pbitem ¿Cuánto tiempo
%         pasa entre dos 
%         Ping de MQTT?
% \end{problemlist}
% 
% Guarde la captura en el fichero
% \texttt{captura-publish-grupoX.pcapng}.
% 
% 
% \subsection*{Desconexión y QoS}
% \noindent A continuación vamos a probar
% los distintos niveles de QoS ofrecidos
% por MQTT. Para ello basta con
% añadir el argumento
% \texttt{qos=$q$} a la función
% \texttt{client.publish()}, donde
% $q\in\{0,1,2\}$ especifica el nivel
% QoS de MQTT.
% 
% En primer lugar vamos a probar
% qué sucede con un Qos de 0 si se
% cae la conexión. Para ello vamos a
% apagar la interfaz
% \texttt{eth0} usando el
% siguiente comando
% \begin{lstlisting}[language=bash]
% $ ip link set down dev eth0
% \end{lstlisting}
% y la encenderemos con el siguiente comando
% \begin{lstlisting}[language=bash]
% $ ip link set up dev eth0
% \end{lstlisting}
% 
% 
% 
% \subsection*{QoS 0}
% \noindent
% Ponga un QoS 0 y ejecute el
% \texttt{client.py} usando el
% siguiente comando para guardar la salida:
% \begin{lstlisting}[language=bash]
% $ python3 client.py 2>&1 | tee /tmp/qos0-grupoX.log
% \end{lstlisting}
% 
% Inicie una
% captura en wireshark filtrando
% tráfico MQTT y espere a que
% se envíen, al menos, dos tramas
% PUBLISH. Apague la interfaz
% \texttt{eth0} y espere a que
% el cliente imprima
% ``Failed to send message [...]''.
% Vuelva a encender la interfaz.
% 
% 
% Responda a las siguientes preguntas
% en el JSON de respuestas:
% \begin{problemlist}
%     \stepcounter{enumi}
%     \stepcounter{enumi}
%     \stepcounter{enumi}
%     \stepcounter{enumi}
%     \pbitem Especifique el \verb|Idx|
%         de las muestras identificadas
%         como perdidas por el cliente.
%     \pbitem Especifique el \verb|Idx|
%         de las muestras que no se
%         han enviado con éxito.
% \end{problemlist}
% Detenga la captura de wireshark y
% guárdela en el fichero
% \texttt{qos0-grupoX.pcapng}.
% 
% 
% \subsection*{QoS 1}
% \noindent
% Ponga un QoS 1 y ejecute el \texttt{client.py}
% como en el apartado anterior, esta vez
% guardando el log en el archivo
% \texttt{qos1-grupoX.log}.
% 
% De nuevo, inicie una captura en wireshark
% filtrando tráfico MQTT y espere a que
% se envíen, al menos, dos tramas PUBLISH.
% Apague la interfaz \texttt{eth0}
% y espere de nuevo a que el cliente
% imprima ``Failed to send message [...]''.
% Vuelva a encender la interfaz.
% 
% 
% 
% 
% 
% \begin{problemlist}
%     \setcounter{enumi}{6}
%     \pbitem Indique el
%         \texttt{Message Identifier}
%         de los mensajes duplicados.
% 
% 
%     \pbitem Repita otra captura
%         incrementando el \texttt{keepalive}
%         al doble e indique el
%         \texttt{Message Identifier}
%         de los mensajes perdidos.
% 
% \end{problemlist}
% Guarde ambas capturas
% en los ficheros
% \texttt{qos1-grupoX.pcapng} y\\
% \texttt{qos1-x2keepalive-grupoX.pcapng}.
% 
% 
% 
% 
% \section*{Entrega}
% \noindent Se subirá a moodle un archivo
% \texttt{clienteX.zip}
% (con \texttt{X} el número de grupo)
% que contenga:
% \begin{enumerate}
%     \item el cliente \texttt{client.py};
%     \item el archivo \texttt{utils.py};
%     \item el JSON de respuestas \texttt{respuestas-X.json};
%     \item las trazas de Wireskark
%         \texttt{captura-connect-grupoX.pcapng},\\
%         \texttt{captura-publish-grupoX.pcapng},
%         \texttt{qos0-grupoX.pcapng},\\
%         \texttt{qos1-grupoX.pcapng},\
%         \texttt{qos1-x2keepalive-grupoX.pcapng}; y
%     \item los logs
%         \texttt{qos0-grupoX.log},
%         \texttt{qos1-grupoX.log}.
% \end{enumerate}
% 
% \begin{tcolorbox}
%     \textbf{Atención I}: las capturas
%     deben contener
%     \emph{solamente} tráfico MQTT.\\
%     \textbf{Atención II}: una entrega
%     sin los archivos especificados,
%     o con archivos sin formato especificado
%     tendrá un 0 en los \textbf{Problemas}
%     correspondientes.
% \end{tcolorbox}


\end{document}
